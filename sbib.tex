%%%%%%%%%%%%%%%%%%%%%%%%%%%%%%%%%%%%%%%%%
% Thin Sectioned Essay
% LaTeX Template
% Version 1.0 (3/8/13)
%
% This template has been downloaded from:
% http://www.LaTeXTemplates.com
%
% Original Author:
% Nicolas Diaz (nsdiaz@uc.cl) with extensive modifications by:
% Vel (vel@latextemplates.com)
%
% License:
% CC BY-NC-SA 3.0 (http://creativecommons.org/licenses/by-nc-sa/3.0/)
%
%%%%%%%%%%%%%%%%%%%%%%%%%%%%%%%%%%%%%%%%%

%----------------------------------------------------------------------------------------
%	PACKAGES AND OTHER DOCUMENT CONFIGURATIONS
%----------------------------------------------------------------------------------------

\documentclass[a4paper, 11pt]{article} % Font size (can be 10pt, 11pt or 12pt) and paper size (remove a4paper for US letter paper)

\usepackage[utf8]{inputenc}
\usepackage[T1]{fontenc}
\usepackage[francais]{babel}

\usepackage[protrusion=true,expansion=true]{microtype} % Better typography
\usepackage{graphicx} % Required for including pictures
\usepackage{wrapfig} % Allows in-line images

\usepackage{amsmath}

\usepackage{mathpazo} % Use the Palatino font
\linespread{1.05} % Change line spacing here, Palatino benefits from a slight increase by default

\makeatletter
\renewcommand\@biblabel[1]{\textbf{#1.}} % Change the square brackets for each bibliography item from '[1]' to '1.'
\renewcommand{\@listI}{\itemsep=0pt} % Reduce the space between items in the itemize and enumerate environments and the bibliography

\renewcommand{\maketitle}{ % Customize the title - do not edit title and author name here, see the TITLE block below
\begin{flushright} % Right align
{\LARGE\@title} % Increase the font size of the title

\vspace{50pt} % Some vertical space between the title and author name

{\large\@author} % Author name
\\\@date % Date

\vspace{40pt} % Some vertical space between the author block and abstract
\end{flushright}
}

%----------------------------------------------------------------------------------------
%	TITLE
%----------------------------------------------------------------------------------------

\title{\textbf{Synthèse Bibliographique}\\ % Title
Les Systèmes de Notations} % Subtitle

\author{\textsc{Brunisholz, Di Folco, Pigeon} % Author
\\{\textit{INSA Lyon}}} % Institution

\date{\today} % Date

%----------------------------------------------------------------------------------------

\begin{document}

\maketitle % Print the title section

%----------------------------------------------------------------------------------------
%	ABSTRACT AND KEYWORDS
%----------------------------------------------------------------------------------------

%\renewcommand{\abstractname}{Summary} % Uncomment to change the name of the abstract to something else

\begin{abstract}
	Ce document de synthèse a pour but de traiter des systèmes de réputation en informatique.
	Il abordera ainsi les systèmes de réputation dans leur généralité en guise d'introduction.
	Sera ensuite présenté un état de l'art des différents algorithmes de notations.
\end{abstract}

%\hspace*{3,6mm}\textit{Keywords:} lorem , ipsum , dolor , sit amet , lectus % Keywords

\vspace{30pt} % Some vertical space between the abstract and first section

%----------------------------------------------------------------------------------------
%	ESSAY BODY
%----------------------------------------------------------------------------------------

\section{Introduction}
Les systèmes de réputation sont à la base de nos prises de décisions, en l'absence de meilleurs informations concernant celles-ci.
Ils ont de ce fait une part extrement importante depuis l'apparition d'internet, car sans donner de métriques concernant la qualité d'un contenu Web,
il serait quasiment impossible pour une personne de trouver un résultat pertinant lors d'une recherche internet, d'estimer le risque avant un achat en ligne,
de s'assurer de la qualité d'un contenu consommé, etc...
Ayant conscience de l'enjeu qu'ils représentent, il est donc intéressant de se pencher sur ceux-ci, afin de comprendre leur fonctionnement.

Ainsi, en guise d'introduction, il sera proposé de voir ce qu'est une réputation, et quelles sont les caractéristiques générales d'un système de réputation.

\subsection{Définition de la réputation}
De manière naïve, la réputation pourrait se réduire au fait que quelqu'un apporte un jugement sur quelque chose~\cite{FarmerGlass2010}.
Le jugement apporté, peut quand à lui être de deux types distinct~\cite{FarmerGlass2010}.
\begin{description}
	\item[Explicite] Le jugement émis peut être directe ou indirecte, mais quoi qu'il en soit, il s'agit d'une affirmation émise par l'auteur du jugement.
	\item[Implicite] Le jugement n'est plus une affirmation, mais un comportement. C'est en effet l'action d'une personne vis à vis d'un sujet qui atteste de la réputation de celui-ci.
\end{description}

Par ailleurs, les systèmes de réputation sont généralement tous fondés autour de trois axes principaux~\cite{HoffmanZageNita2007} : la Formulation, l'Application et la Dispersion.

\subsubsection{La Formulation}
C'est au cours de cette étapes que l'algorithme de calcul de réputation est enoncé dans sa formulation théorique.
Plus précisement, c'est au cours de cette phase que sont exprimés comment seront colléctée les données utilisées, ainsi que leurs type.
Enfin, il est précisé sous quelle forme l'algorithme rendra son résultat. Celui-ci peut être binaire, discret ou continu.

\subsubsection{L'Application}
Il s'agit de l'implémentation de l'algorithme.
Contrairement à la Formulation, cette étape prends en compte les différentes contraintes du milieu dans lequel l'algorithme va être appliqué.
Il s'agit alors de respecter la métrique voulue, tout en essayant d'être resistant aux attaques malicieuses.
Ainsi il est notamment discuté des méthodes de stockage et de calculs qui vont être utilisées.

\subsubsection{La Dispersion}
La Dispersion correspond aux méthodes employées afin de transmettre l'information relative à la réputation d'une entité aux différents acteurs concernés.
Il sera alors discuté de la manière dont la réputation est stockée, comment celle-ci est transmise, et sa redondance.

\section{Etat de l'art}
Il s'agit dans cette partie de voir quels sont les différents systèmes de réputations actuels, et de les comparer.

\subsection{WikiTrust}
Il s'agit de l'algorithme utilisé par Wikipedia pour évaluer la réputation des auteurs et des articles. Il repose sur l'analyse des modifications de contenu d'un article.
En effet, si une personne modifie un article sur l'encyclopédie, la réputation de l'article varie de la manière suivante~\cite{WikiTrust2010}~\cite{WikiTrustSite} :
ce qui a été reécrit gagne un petit montant de réputation pondéré par la réputation de l'auteur,
et la partie restée inchangée gagne elle aussi en réputation en fonction de la réputation du modificateur.
L'indice de confiance global de l'article est ensuite reévalué.

La réputation d'un auteur est calculé d'une manière similaire, à savoir que lorsqu'un article est modifié,
la réputation de l'auteur original est elle aussi modifié en fonction de la quantité de contenu modifié, pondérée par la réputation du modificateur.

Enfin, WikiTrust part du postulat que les personnes malveillante possède pas ou peu de réputations, ce qui rends l'algorithme de calcul de réputation resistant à l'auto promotion~\cite{Tulungan2013}.

\subsection{Système de Feedback}
L'idée principale de cette méthode est de permettre d'évoluer un ou plusieurs acteurs en fonction d'une action à posteriori.
Lorsqu'une action est réalisée, les parties prenantes se notent les unes les autres et la réputation des différents acteurs devient la somme des avis des acteurs ayant déjà eu des interactions avec celui-ci.

Si nous prennons le cas d'Ebay qui a mis en application ce principe, leur implantation est réalisé de la manière suivante : une fois la vente conclue, le vendeur et l'acheteur, les deux parties perenante de la transaction, se voient demander de s'évaluer mutuellement sous les 60 jours.
Les évaluations peuvent avoir les valeur suivante : +1 (la transaction s'est très bien déroulée), 0 (la transaction s'est déroulée correctement, mais sans plus), -1 (la transaction s'est mal déroulée).
Ensuite la somme des évaluations est effectuée ce qui donne la réputation d'un membre.

eBay rajoute cependant plusieurs contraintes afin d'éviter les débordements.
La première variante est que lorsque plusieurs transactions entre un même vendeur et un même acheteur sont effectués la même semaine, on ne répercute pas immédiatement la réputation de chacun mais on attend la fin de la semaine, puis on regarde s'il y a plus de votes positif ou négatif et enfin on ajoute un point à la réputation de l'acteur si son bilan hebdomadaire est positif ou on retranche un point de sa réputation s'il est négatif.
La seconde variation est une protection des vendeurs importants ayant plus d'un an d'ancienneté : les membres ne peuvent pas laisser de vôte négatif ou neutre avant une semaine.

Ces modifications sont là pour empêcher des actions de décrédibilisation, en revanche rien ne protège d'une fausse amélioration de la réputation.

\subsection{Karma}
\subsubsection{Concept}
Le Karma permet d'évaluer la réputation uniquement d'une personne au sein d'une communauté Web. Le nombre de points Karma est attribué en fonction :
\begin{itemize}
	\item De la vieillesse de l'utilisateur au sein de la communauté
	\item Des actions auxquelles il a participé
	\item Des résultats qu'il a obtenus
	\item De ce que pense les autres utilisateurs
\end{itemize}
Inclure le Karma dans le système de réputation d'un site Web permet de rendre celui-ci plus performant et plus proche de la réalité. Il est d'ailleurs rarement utilisé seul pour cacluler une réputation mais associé à d'autres algorithmes.

Il existe deux formes de Karma et une combinaison :
\begin{description}
	\item[Participation] calcule le Karma en fonction de la participation du membre de a communauté
	\item[Qualité] calcule le Karma en fonction de la qualité des contributions du membre
	\item[Robuste] est une combinaison des deux précédentes
\end{description}

\subsubsection{Calcul du Karma, modèle de participation}
La calcul de la réputation grâce à ce modèle se fait via l'attribution de points. Lorsqu'un membre effectue une action au sein de la communauté, un nombre de points lui est attribué en fonction de cette action. La réputation de ce membre est donc la somme des points obtenus.

Il existe également, un système de points négatifs. Lorsqu'un membre effectue une mauvaise action, on lui attribue un nombre de points négatifs correspondant à cette mauvaise action. Le système est le même que celui des points positifs, excepté qu'un score élevé correspond à une mauvaise réputation.

\subsubsection{Calcul du Karma, modèle de qualité}
Dans ce modèle, le nombre de contributions n'a pas d'importance. Seule compte la qualité de ces dernières. 

Le modèle de Feedback utilisé par eBay décrit précédemment est un exemple de modèle de qualité du Karma.

\begin{figure}
	\begin{center}
		\includegraphics[width=12cm]{karma.png} 
	\end{center}
	\caption{Modèle robuste~\cite{FarmerGlass2010}}
	\label{robusteKarma}
\end{figure}

\subsubsection{Calcul du Karma, modèle robuste}
En combinant les deux modèles précédents, on obtient le modèle robuste, comme le montre la figure~\ref{robusteKarma}. 

\subsubsection{Reddit.com}
Reddit.com est un site communautaire qui permet à chaque membre de soumettre des liens aux autres membres qui peuvent ensuite voter pour ce lien. Les membres peuvent également commenter les liens proposer et voter pour ou contre les commentaires.

A chaque membre est associé un "Karma". Ce dernier est calculé en fonction des votes des membres. Lorsqu'un membre vote pour un lien proposé, le karma du membre qui a proposé le lien est augmenté d'un point. Lorsqu'un membre vote contre un lien proposé, le karma du membre qui a proposé le lien est diminué d'un point. Si le membre a obtenu plus de votes négatifs que de votes positifs, son karma est donc négatif. Les liens proposés par des membres qui ont un bon karma sont plus facilement visibles par les autres membres, que les liens proposés par des membres qui ont un moins bon karma~\cite{RedditTroll}.

De même, le karma d'un membre est augmenté d'un point lorsqu'un autre membre vote pour un de ses commentaires et est diminué d'un point lorsqu'un autre membre vote contre~\cite{RedditTroll}.

\subsubsection{Comment contrôler le karma ?}
Expliquer comment reddit tente de contrer les trolls ?

\subsection{Tulungan}
Il s'agit d'un algorithme colaboratif (user-driven) permettant de donner une métrique sur le contenu d'une URL, en lui associant une ou plusieurs catégories d'URLs.
L'algorithme part du postulat qu'un utilisateur peut à la fois contribuer à la classification de contenu, et noter du contenu.
Il s'éxécute en trois phase : la phase d'Initialisation, la phase de Contribution et Notation, et la phase de Calcul~\cite{Tulungan2013}.

La phase d'Initialisation a lieu à chaque fois qu'un nouvel uilisateur apparaît, ou qu'une nouvelle catégorie d'URL est crée.
Ainsi les valeurs de contributions et de notations sont initialisées avec une valeur proche de zéro.

La phase de contribution et de notation permet à chaque participants de prendre part à la catégorisation en exprimant si un contenu web appartient ou non à différentes catégories.
Par ailleurs au cours de cette phase seront choisit des noteurs potentiels (car ils sont susceptibles de donner une mauvaise métrique) en fonction de leurs précédentes contributions et notations.
Ceci a pour but de séléctionner des personnes en rapport avec le contenu qu'il leurs sera demandé de noter.
Dernière étape de cette phase, la notation à proprement parler. Les participants doivent alors noter trois URLs en exprimant si elles sont bien catégorisées.
Deux des URLs sont des URLs de contrôle, dont les catégories d'appartenances sont bien connues.
La dernière est quand à elle l'URL dont on souhaite déterminer si la catégorisation est pertinente ou non.
Les trois URLs sont présentées de manière à ce que l'utilisateur ne sache pas laquelle est la réelle cible de l'algorithme.
L'algorithme considère alors que si un utilisateur réponds correctement aux deux URLs connu, alors il est probable que celui-ci ai répondu de manière sérieuse à la troisième.
Cette approche peut être comparée aux Captcha.

La dernière phase de l'algorithme consiste à reévaluer l'ensemble des réputations des URLs, des noteurs et des contributeurs.

Pour finir sur Tulungan, il est interessant de noter que celui-ci est un algorithme "consensus-independant", car il n'a besoins que de 20\% de participants honnête pour etre efficace (contrairement à des algorithmes plus classiques necessitant un proportion de participants honnête de l'ordre de 50\%),
et est resistant contre le "self-promoting" (modélisé par une attaque Sybile)~\cite{Tulungan2013}.

\subsection{Système de Digg}
Digg est un site web de partage d'article communautaire, un membre lit un article sur l'internet ou écrit un article sur un espace personnel, puis il le soumet à Digg.
Ensuite les autres membres du site votent pour ou contre cet article et s'il obtient assez de voix il est publié en page d'accueil de Digg.

L'algorithme est donc simple : il n'y a pas de votes négatif, la réputation d'un article ne peut que croître, une fois le nombre de voix critique atteint, il est publié en première page, si non, au bout d'un certain temps, l'article est supprimé des propositions.

L'un des principaux avantages est que ce système ne permet pas de discréditation et qu'il n'attise pas l'augmentation artificiel car l'enjeu est moindre, il s'agit d'une question de temps.
En revanche il ne permet pas de réprésentation explicite du désaccord des lecteurs, pour cela il faudrait opposer le nombre de lecteur et le nombre de votants.

\subsection{}

%------------------------------------------------

\bibliographystyle{plain}
\bibliography{biblio}

\end{document}
